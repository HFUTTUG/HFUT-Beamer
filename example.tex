% !TEX TS-program = xelatex
% !TEX encoding = UTF-8

\documentclass[aspectratio=169]{beamer}
\usetheme{hfut}

\usepackage{amsmath,amsfonts,amssymb}
\usepackage{blindtext}
\usepackage{multirow}
\usepackage{booktabs}
\usepackage{layout}

%% 解注以获得中文首行缩进,不建议全局使用!
%% 正确使用方法如之后的frame所示
% \setlength{\parskip}{6pt}
% \setlength{\parindent}{2em}

\title{基于×××××的设计与实现}
\subtitle{类型:论文}
\author[{Fw[a]rd}]{HFUT \TeX{} User Group \and Fw[a]rd}
\institute{HFUT \TeX{} User Group}
\date{May 30\textsuperscript{th} 2021}

\begin{document}

\begin{frame}
	\maketitle
\end{frame}

\section*{\S}
\begin{frame}{目录}
	\tableofcontents
\end{frame}

\section{研究背景和现状}

\begin{frame}{项目背景}
    \framesubtitle{文章生成器生成}
{
    \setlength{\parskip}{6pt}
    \setlength{\parindent}{2em}
模板因何而发生?经过上述讨论,可是,即使是这样,模板的出现仍然代表了一定的意义。这样看来,本人也是经过了深思熟虑,在每个日日夜夜思考这个问题。

既然如此,一般来说,就我个人来说,模板对我的意义,不能不说非常重大。总结的来说,王阳明说过一句著名的话,故立志者,为学之心也;为学者,立志之事也。这句话把我们带到了一个新的维度去思考这个问题: 我们都知道,只要有意义,那么就必须慎重考虑。 
}
\end{frame}

\section{英文介绍}
\begin{frame}
\frametitle{Bullet Points}
\vspace{-0.3cm}
\begin{itemize}[<+->]
	\item Most engineers are lazy ... and that is often a good thing
	\begin{itemize}[<+->]
		\item (\textit{lazy} $=$ \textit{to do things in the most efficient way})
	\end{itemize}
	\item Engineers are terrible story tellers ... they prefer content to form
	\item Readers are lazy ... need self contained and easy to read material
	\item \LaTeX{} can help
\end{itemize}
\end{frame}

\begin{frame}[fragile] % Need to use the fragile option when verbatim is used in the slide
\frametitle{Verbatim}
\begin{example}[Theorem Slide Code]
\begin{verbatim}
\begin{frame}
\frametitle{Theorem}
\begin{theorem}[Mass--energy equivalence]
$E = mc^2$
\end{theorem}
\end{frame}\end{verbatim}
\end{example}
\end{frame}


\begin{frame}
\frametitle{\TeX{}}
\begin{columns}
\begin{column}{4cm}
	\begin{figure}
		\includegraphics[height=3cm]{images/HFUT_badge.pdf}
	\end{figure}
	\begin{center}
		\tiny
		合肥工业大学 \\
	\end{center}
\end{column}
\begin{column}{6cm}
	\begin{itemize}
		\item \TeX{} was created by Donald Knuth in 1978
		\item A typesetting macro language and compiler:
		\begin{itemize}
			\item Readable mathematics
			\item Better hyphenation
			\item Optimized justification
			\item Font management tools
			\item Cross-compatibility
		\end{itemize}
		\item Code -- Compile -- Visualize
	\end{itemize}
\end{column}
\end{columns}
\end{frame}



\section{Data and Method}

\begin{frame}
\frametitle{Figure}
\vspace{-0.3cm}
\begin{figure}[h]
\centering
\includegraphics[width=0.3\textwidth]{images/HFUT_badge.pdf}
\caption{合肥工业大学的Logo}
\end{figure}
\end{frame}


\begin{frame}
\frametitle{Editors and Compilers}
\begin{itemize}
\item To install in your machine
\begin{itemize}
\item Check \texttt{latex-project.org}
\end{itemize}
\item In the cloud
\begin{itemize}
\item ShareLatex : \texttt{www.sharelatex.com}
\item Overleaf : \texttt{www.overleaf.com}
\end{itemize}
\end{itemize}
\vskip 1cm
\begin{block}{Overleaf}
    Overleaf includes a history of all of your changes so you can see exactly who changed what, and when. 
\end{block}
\end{frame}


\begin{frame}
\frametitle{Multiple Columns}
\begin{columns}[c]
\column{.45\textwidth} % Left column and width
\textbf{Heading}
\begin{enumerate}
\item Statement
\item Explanation
\item Example
\end{enumerate}
\column{.5\textwidth} % Right column and width
Lorem ipsum dolor sit amet, consectetur adipiscing elit. Integer lectus nisl, ultricies in feugiat rutrum, porttitor sit amet augue. Aliquam ut tortor mauris. Sed volutpat ante purus, quis accumsan dolor.
\end{columns}
\end{frame}



\section{Conclusion}

\begin{frame}
\frametitle{Table and Equation}
\vspace{-0.3cm}
\begin{table}
\caption{Table caption}
\begin{tabular}{l l l}
\toprule
\textbf{Treatments} & \textbf{Response 1} & \textbf{Response 2}\\
\midrule
Treatment 1 & 0.0003262 & 0.562 \\
Treatment 2 & 0.0015681 & 0.910 \\
Treatment 3 & 0.0009271 & 0.296 \\
\bottomrule
\end{tabular}
\end{table}

\begin{equation} % use equation* to remove equation number
\label{eq:matrix_transpose}
\left[
\begin{array}{ccc}
a_{11} & \cdots & a_{1n} \\
\vdots & \ddots & \vdots \\
a_{n1} & \cdots & a_{nn}
\end{array}
\right]^T
=
\left[
\begin{array}{ccc}
a_{11} & \cdots & a_{n1} \\
\vdots & \ddots & \vdots \\
a_{1n} & \cdots & a_{nn}
\end{array}
\right]
\end{equation}
\end{frame}


\begin{frame}
\frametitle{References}
\footnotesize{
\begin{thebibliography}{99} % Beamer does not support BibTeX so references must be inserted manually as below
\bibitem[Smith, 2012]{p1} John Smith (2012)
\newblock Title of the publication
\newblock \emph{Journal Name} 12(3), 45 -- 678.
\end{thebibliography}
}
\end{frame}

\begin{frame}
谢谢各位!
\end{frame}


\end{document}
