% !TEX TS-program = xelatex
% !TEX encoding = UTF-8

\documentclass[aspectratio=169]{beamer}
\usetheme{hfut-sx}

\usepackage{amsmath,amsfonts,amssymb}
\usepackage{blindtext}
\usepackage{multirow}
\usepackage{booktabs}
\usepackage{layout}
\usepackage{minted}
\usepackage[natbib=true]{biblatex}
\addbibresource{example.bib}

\setbeamertemplate{frametitle continuation}[from second]

\AtBeginSection[]
{
  \begin{frame}{目录}
    \tableofcontents[currentsection]
  \end{frame}
}



\newcommand{\BibTeX}{\textsc{Bib}\TeX{}}
\newcommand{\BibLaTeX}{\textsc{Bib}\LaTeX{}}
\newcommand{\Beamer}{\textsc{Beamer}}
\newcommand{\enableindent}{\setlength{\parskip}{6pt}\setlength{\parindent}{2em}}
% algorithm
\usepackage{algorithm}
\usepackage{algorithmicx,algpseudocode}
\floatname{algorithm}{算法}

\title{HFUT-Beamer使用指南}
\subtitle{以及个人的一些废话}
\author{Fw[a]rd}
\institute{HFUT \TeX{} User Group}
\date{\today}

\begin{document}

\begin{frame}
	\maketitle
\end{frame}

\begin{frame}{目录}
	\tableofcontents
\end{frame}

\section{\Beamer{}简单入门}

\begin{frame}[fragile]{标题}{和副标题}
一帧幻灯片的标题可以用如下方式指定:
\begin{itemize}
	\item \mintinline{tex}{\begin{frame}{标题}{副标题}}
	\item \mintinline{tex}{\frametitle{标题}}
	\item \mintinline{tex}{\framesubtitle{副标题}}
\end{itemize}

\end{frame}

\begin{frame}{Block环境}

Block环境是\Beamer{}的一个特色功能,可以用来引导读者视线,突出幻灯片中的重点内容。

\begin{block}{这是Block的标题}
	这是Block的内容
\end{block}

\begin{theorem}<1->
There exists an infinite set.
\end{theorem}

\begin{proof}<2->
This follows from the axiom of infinity.
\end{proof}

\begin{example}<3->[Natural Numbers]
The set of natural numbers is infinite.
\end{example}

\end{frame}

\begin{frame}[fragile]{排版}
	\enableindent

	很多MS Powerpoint文档拥有复杂的排版,\Beamer{}其实也可以做到类似的效果。
	\begin{columns}
		\begin{column}{0.45\textwidth}
			比如说,\texttt{columns}环境就可以提供多栏的排版格式,这允许你进行左右对照。
		\end{column}
		\begin{column}{0.45\textwidth}
\begin{minted}{tex}
\begin{columns}
    \begin{column}{0.45\textwidth}
        % ...
    \end{column}
    \begin{column}{0.45\textwidth}
        % ...
    \end{column}
\end{columns}
\end{minted}
		\end{column}
	\end{columns}
\end{frame}

\begin{frame}[fragile]{幻灯片脚注}
    \setbeamerfont{footnote}{size=\tiny}

\begin{itemize}
	\item 脚注用\mintinline{tex}{\footnote}命令\footnote{这是一个脚注}。
	\item 脚注引用用\mintinline{tex}{\footcite}命令\footcite[需要\BibLaTeX{},下同]{biblatex}。
	\item 脚注完整引用用\mintinline{tex}{\footfullcite}命令\footfullcite{biblatex}。
\end{itemize}
\end{frame}

\begin{frame}{图片、表格、算法}
\enableindent
	这些浮动体环境在\Beamer{}下并无什么不同,我们可以通过正常的方法放置这些环境。但是有时候(特别是算法环境)放置策略需要采用\texttt{[H]}。

	\only<1>{
		\begin{figure}[htb]
			\centering
			\includegraphics[width=0.2\textwidth]{images/HFUT_badge.pdf}
			\caption{合肥工业大学校徽}
		\end{figure}
	}
	\only<2>{
		\begin{table}
		\caption{Table caption}
		\begin{tabular}{l l l}
		\toprule
		\textbf{Treatments} & \textbf{Response 1} & \textbf{Response 2}\\
		\midrule
		Treatment 1 & 0.0003262 & 0.562 \\
		Treatment 2 & 0.0015681 & 0.910 \\
		Treatment 3 & 0.0009271 & 0.296 \\
		\bottomrule
		\end{tabular}
		\end{table}
	}
	\only<3>{
\begin{algorithm}[H]
\caption{神经网络训练过程}
\begin{algorithmic}[1]
\State 初始化网络权值 $\mathcal{W}$
\For{$t=0 \to max\_train\_steps$}
\State {\bfseries 输入}一批次$x$以计算$L(t)$
\State 计算梯度$\nabla_{\mathcal{W}} L(t)$
\State 使用梯度$\nabla_{\mathcal{W}} L(t)$更新$\mathcal{W}(t) \mapsto \mathcal{W}(t+1)$
\EndFor
\end{algorithmic}
\end{algorithm}
	}
\end{frame}

\section{动画}

\begin{frame}{简单动画}
	\Beamer{}比较方便的是实现一些简单的动画效果。
	
	比方说:
	\begin{itemize}
		\item 单击显示/隐藏
  		\item 单击变色
	\end{itemize}
\end{frame}

\section{引用}

\begin{frame}{引用最好用\BibLaTeX{}}
	其实\BibTeX{}不一定不可以在\Beamer{}中使用,但是它在\Beamer{}下很可能不如\BibLaTeX{}方便。
\end{frame}

\section{\Beamer{}的一些坑}

\begin{frame}{\texttt{[fragile]}选项}
	\enableindent

	如果帧包含脆性文本(fragile text),在排版帧时将使用不同的内部机制(mechanisms)以确保在帧
内重置字符代码(reset character codes)。此时需要给\texttt{frame}添加\texttt{[fragile]}选项。

	所谓的脆性文本通常包括:
	\begin{itemize}
		\item \texttt{verbatim}环境
		\item \texttt{lstlisting}环境
  		\item \texttt{minted}环境
	\end{itemize}
\end{frame}

\begin{frame}{打印参考文献列表需要特殊处理}
	
	参考文献列表通常比较长,一帧幻灯片通常无法满足其空间需求。这时候需要给\texttt{frame}环境添加\texttt{[allowframebreaks]}选项。
\end{frame}

\begin{frame}[allowframebreaks]
\frametitle{参考书目}
{
	\tiny
	\nocite{*}
	\printbibliography[heading=none]
}
\end{frame}

\begin{frame}
	欢迎进行贡献和加入我们!
\end{frame}

\end{document}
